The mean, also known as the \textbf{arithmetic mean} or \textbf{average}, is obtained by dividing the sum of observed values, data values, by the number of observations. This gives a good idea as to what the points as closest to. Occasionally a value that differs greatly can be seen, the mean would incorporate the occasional outlying data. This means the average is susceptible to outlying values. The idea is to find the center of the data set. It is the most common type of average.

The \textbf{sample mean} is represented by an x with an over line. The numbers of samples is represented by the letter $n$.

\begin{listequbox}
  {\bar{X} = \dfrac{1}{n} \mathlarger{\sum\limits_{i=1}^n x_i}}{equmeansmp}{Sample mean}
\end{listequbox}

The \textbf{population mean} is represented by the greek letter mu, $\mu$. The total number of values is represented by the letter $N$.

\begin{listequbox}
  {\mu = \dfrac{1}{N} \mathlarger{\sum\limits_{i=1}^N x_i}}{equmeanpop}{Population Mean}
\end{listequbox}

Both equations \ref{equmeansmp} and \ref{equmeanpop} are the same, the only change are the symbols employed. They can be used when the error associated with each measurement is the same or unknown.
