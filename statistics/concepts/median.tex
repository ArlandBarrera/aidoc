The median is the middle value of a set of data containing an odd number of values, or the average of the two middle values of a set of data with an even nummber of values. The median is specially helpful when separating data into two equal sized distributions. It is useful if there is an interest in knowing the range of values the system could be be operating in. Half the values should be above and half the values should be below, so there is an idea of where the middle operating point is. Because of this the median less sentitive to outlier data.

For \textbf{odd numbers} the median is the data points $n$ plus 1 divided by 2.

\begin{listequbox}
  {\text{Median} = \dfrac{n + 1}{2}}{equmedianodd}{Odd median}
\end{listequbox}

For \textbf{even numbers} the median is the average of the data points $n$ divided by two, and $n$ divided by two plus one.

\begin{listequbox}
  {\text{Median} = \dfrac{1}{2} \left(\dfrac{n}{2} + \left(\dfrac{n}{2} + 1 \right)\right)}{equmedianeven}{Even median}
\end{listequbox}
